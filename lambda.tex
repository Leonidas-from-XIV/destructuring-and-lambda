\documentclass{beamer}
\usepackage{fontspec}
\usepackage{minted}
\usepackage{tango}
\usepackage{ccicons}
\usepackage{tikz}
\usepackage[normalem]{ulem}
\usepackage{csquotes}
\usepackage{amssymb}
\usepackage{scrextend}

\mode<presentation>{\usetheme{rwo}}

\setsansfont{Yanone Kaffeesatz Regular}
\setmainfont{EB Garamond}
\setmonofont[Scale=0.8]{Droid Sans Mono Dotted}
\changefontsizes{15pt}

\usemintedstyle{tango}
\setbeamerfont{title}{family=\rmfamily}

\setlength{\parskip}{8pt plus 1pt minus 1pt}

\defbeamertemplate*{title page}{rwo}[1][]
{
  \vbox{}
  \vfill
  \begin{centering}
    \begin{beamercolorbox}[sep=8pt,center,#1]{title}
      \usebeamerfont{title}\inserttitle\par%
      \ifx\insertsubtitle\@empty%
      \else%
        \vskip0.25em%
        {\usebeamerfont{subtitle}\usebeamercolor[fg]{subtitle}\insertsubtitle\par}%
      \fi%
    \end{beamercolorbox}%
    \vskip1em\par
    \begin{beamercolorbox}[sep=8pt,center,#1]{author}
      \usebeamerfont{author}\insertauthor
    \end{beamercolorbox}
    \begin{beamercolorbox}[sep=8pt,center,#1]{institute}
      \usebeamerfont{institute}\insertinstitute
    \end{beamercolorbox}
    \begin{beamercolorbox}[sep=8pt,center,#1]{date}
      \usebeamerfont{date}\insertdate
    \end{beamercolorbox}\vskip0.5em
    {\usebeamercolor[fg]{titlegraphic}\inserttitlegraphic\par}

    \begin{center}
      \ccby{}\\
      {\tiny This work is licensed under a
      \href{http://creativecommons.org/licenses/by/3.0/deed.en_US}%
      {Creative Commons Attribution 3.0 Unported License}.}
    \end{center}
    \vspace*{-2.5ex}
  \end{centering}
  \vfill
}

\renewcommand{\example}[1]{{\usebeamercolor[fg]{example text} #1}}
\renewcommand{\ULthickness}[0]{0.23ex}

\definecolor{cljbackground}{HTML}{e9e2cb}
\definecolor{cljtext}{HTML}{d01b24}
\setbeamercolor{background canvas}{bg=cljbackground}
\setbeamercolor{normal text}{fg=cljtext}


\title{Let over Lambda}
\date{13.~October~2013}

\begin{document}

\frame{\titlepage}

\centerframe{
  What is a lambda?
}

\begin{frame}[fragile]
  \begin{center}
    A lambda is a function.
  \end{center}
  \pause
  \begin{minted}[gobble=4]{clojure}
    (fn [x] (* x 2))
  \end{minted}
\end{frame}

\begin{frame}[fragile]
  \begin{center}
    But why the hell "lambda"?
  \end{center}
  \begin{minted}[gobble=4]{scheme}
    (lambda (x) (* x 2))
  \end{minted}
  \begin{center}
    Scheme calls it that. Also: lambda calculus.
  \end{center}
\end{frame}

\begin{frame}[fragile]
  What are scopes?

  \pause
  The bindings (assignments) a function can "see":

  \begin{minted}[gobble=4]{clojure}
    (fn [a] b)
  \end{minted}

  \texttt{a} is "in scope", \texttt{b} maybe.
\end{frame}


\begin{frame}[fragile]
  \begin{minted}[gobble=4]{clojure}
    (fn [a]
      (fn [b]
        (+ a b)))
  \end{minted}
  Takes \texttt{a} and returns a function. A FunctionFactory if you please!
  The returned function takes \texttt{b} and returns \texttt{a} + \texttt{b}
  as result.
\end{frame}

\begin{frame}[fragile]
  This can be useful if you want to preconfigure a function:
  \begin{minted}[gobble=4]{clojure}
    (defn addition [a]
      (fn [b]
        (+ a b)))
    (def increment (addition 1))
    (map increment [1 2 3 4 5])
    ;=> (2 3 4 5 6)
  \end{minted}
\end{frame}

\begin{frame}
  Why should I want this?
  \begin{itemize}
    \item Ever wrote a callback function? They often expect fixed arities
    \item Or just a higher order function like \texttt{map}?
  \end{itemize}
\end{frame}

\begin{frame}[fragile]
  A hypothetical example GUI API
  \begin{minted}[gobble=4,linenos]{clojure}
    (let [address
          (get-postal-address-of :santa)]
      (button :text "Send to santa"
              :onclick
               (fn [event widget]
                 (send-mail-to address))])
  \end{minted}
  Function in line 5 has access to \texttt{address}, but only takes
  \texttt{event} and \texttt{widget} as arguments.
\end{frame}

\begin{frame}[fragile]
  What is this exactly?
  \begin{minted}[gobble=4]{clojure}
    (addition 1)
  \end{minted}
  This returns a function, a so-called "closure". It "closes over" a scope, so
  it knows that \texttt{a} is \texttt{1}.
\end{frame}

\centerframe{Closure}
\centerframe{Clojure}
\centerframe{I see what you did there, Rich Hickey.}

\begin{frame}[fragile]
  Preconfigure a function which does not return a closure? \texttt{partial}
  to the rescue!
  \begin{minted}[gobble=4]{clojure}
    ; old and busted ;-)
    (def increment #(+ 1 %))
    ; new hotness
    (def increment (partial + 1))
    (map increment [1 2 3 4 5])
    ;=> (2 3 4 5 6)
  \end{minted}
  It takes a function (\texttt{+}) and some arguments (\texttt{1}) and
  returns a closure with the arguments pre-set.
\end{frame}

\centerframe{Other fun things with lambdas and scopes}

\centerframe{Dynamic scoping with \texttt{binding}}

\centerframe{Data access control

  \pause
  As we'll implement in the exercises.
}

\end{document}
