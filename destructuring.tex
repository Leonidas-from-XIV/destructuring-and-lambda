\documentclass{beamer}
\usepackage{fontspec}
\usepackage{minted}
\usepackage{tango}
\usepackage{ccicons}
\usepackage{tikz}
\usepackage[normalem]{ulem}
\usepackage{csquotes}
\usepackage{amssymb}
\usepackage{scrextend}

\mode<presentation>{\usetheme{rwo}}

\setsansfont{Yanone Kaffeesatz Regular}
\setmainfont{EB Garamond}
\setmonofont[Scale=0.8]{Droid Sans Mono Dotted}
\changefontsizes{15pt}

\usemintedstyle{tango}
\setbeamerfont{title}{family=\rmfamily}

\setlength{\parskip}{8pt plus 1pt minus 1pt}

\defbeamertemplate*{title page}{rwo}[1][]
{
  \vbox{}
  \vfill
  \begin{centering}
    \begin{beamercolorbox}[sep=8pt,center,#1]{title}
      \usebeamerfont{title}\inserttitle\par%
      \ifx\insertsubtitle\@empty%
      \else%
        \vskip0.25em%
        {\usebeamerfont{subtitle}\usebeamercolor[fg]{subtitle}\insertsubtitle\par}%
      \fi%
    \end{beamercolorbox}%
    \vskip1em\par
    \begin{beamercolorbox}[sep=8pt,center,#1]{author}
      \usebeamerfont{author}\insertauthor
    \end{beamercolorbox}
    \begin{beamercolorbox}[sep=8pt,center,#1]{institute}
      \usebeamerfont{institute}\insertinstitute
    \end{beamercolorbox}
    \begin{beamercolorbox}[sep=8pt,center,#1]{date}
      \usebeamerfont{date}\insertdate
    \end{beamercolorbox}\vskip0.5em
    {\usebeamercolor[fg]{titlegraphic}\inserttitlegraphic\par}

    \begin{center}
      \ccby{}\\
      {\tiny This work is licensed under a
      \href{http://creativecommons.org/licenses/by/3.0/deed.en_US}%
      {Creative Commons Attribution 3.0 Unported License}.}
    \end{center}
    \vspace*{-2.5ex}
  \end{centering}
  \vfill
}

\renewcommand{\example}[1]{{\usebeamercolor[fg]{example text} #1}}
\renewcommand{\ULthickness}[0]{0.23ex}

\definecolor{cljbackground}{HTML}{e9e2cb}
\definecolor{cljtext}{HTML}{d01b24}
\setbeamercolor{background canvas}{bg=cljbackground}
\setbeamercolor{normal text}{fg=cljtext}


\title{Destructuring}
\author{Marek~Kubica}
\date{13.~October~2013}
\institute{Clojure Workshop}

\begin{document}

\frame{\titlepage}

\centerframe{
  Apart from simple types, there are collections, which contain other things.
}

\centerframe{
  But sometimes we want simple data out of a collection type.
}

\centerframe{
  Actually, quite often.
}

\begin{frame}[fragile]
  \begin{minted}[gobble=4]{clojure}
    (def my-collection [1 2 [3 4] 5])
    (first my-collection)
    ;=> 1
    (nth my-collection 2)
    ;=> [3 4]
  \end{minted}
\end{frame}

\begin{frame}[fragile]
  \begin{center}
    But if we want multiple elements this gets ugly fast
  \end{center}
  \begin{minted}[gobble=4]{clojure}
    (let [a (nth my-collection 0)
          b (nth my-collection 1)
          c (nth my-collection 2)
          d (nth my-collection 3)]
      (println a c))
  \end{minted}
\end{frame}

\begin{frame}[fragile]
  \begin{center}
    How many of you know Python?
  \end{center}
  \pause
  \begin{minted}[gobble=4]{python}
    a, b, c, d = my_collection
  \end{minted}
\end{frame}

\centerframe{
  We want this too!
}

\begin{frame}[fragile]
  \begin{center}
    Hey, we can do this too!
  \end{center}
  \begin{minted}[gobble=4]{clojure}
    (let [[a b c d] my-collection]
      (println a c))
  \end{minted}
\end{frame}

\end{document}
